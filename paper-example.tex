\documentclass[dvips]{article}
\usepackage{graphicx}
 
\begin{document}

\title{How to write a paper using LaTeX}
 
\author{Dan Stump}
 
%-----------------------------------------------------------------------
% If your printer does not reproduce dimensions exactly, it may be
% necessary to remove the % signs and adjust the dimensions in the
% following commands:
%
%     \setlength{\textheight}{24cm}
%     \setlength{\textwidth}{16cm}
%
% Similarly for the following, if you need to adjust the positioning
% on the paper:
%
%     \setlength{\topmargin}{-1cm}
%     \setlength{\oddsidemargin}{0pt}
%     \setlength{\evensidemargin}{0pt}
%------------------------------------------------------------------------
 
\maketitle % this produces the title block
 
\begin{abstract}
If you use this template and follow the instructions therein,
your will be able to write a paper using LaTeX.
\end{abstract}
 
\section{INTRODUCTION}

Here give an introduction to your research.

\section{TEXTS}
 
\subsection{Abstract}

Your contribution should  be preceded by a {\it short} Abstract
of not more than 150 words,
written as a single paragraph, as above.

\subsection{Formulae}

Symbols of variables should  be typed in math mode.
For example, to get the Greek letter alpha,
type $\backslash{alpha}$; it should come out looking
like $\alpha$.

Equations should be numbered consecutively, with the 
number enclosed in parentheses and placed flush with
the right-hand margin.

Here is an example of an equation, The Schroedinger
equation for the wave function $\psi(\vec{x},t)$
\begin{equation}\label{Schroedinger}
-\frac{\hbar^{2}}{2m}\nabla^{2}\psi+V\psi=E\psi.
\end{equation}

If you want to refer to the equation, you can use
the {\tt label}, with ($\backslash$ref\{label\}).
For example, the above equation is (\ref{Schroedinger}).

\subsection{Footnotes}

If unavoidable, footnotes should be placed at the bottom
of the page in which they are referred to.

\subsection{References}

References should be cited  in the text using numbers within
square brackets: `example [1],
example [1, 2], example [1--5]', or alternatively as
`Ref. [1], Refs. [1] to [5]', and
`Reference' in full if this word occurs at the
beginning of a sentence.
They should {\it appear in consecutive numerical order}
and should be listed at the end of the text.
Unless you are near the bottom of the last page of text,
do {\it not} start a new page for the list of
references, but continue on the same page.
Note that in the list of references it is
unnecessary to state the title of an article or chapter in
proceedings or in a collection of papers unless a page number
cannot be quoted, e.g. for future publications.

For example, here's a reference to the paper
listed in teh bibliography (at the end):
Ref.\ \cite{Stump}.

\subsubsection{Figures}

All figures should be quoted in consecutive numerical order
in the text and should, for example, be referred to as
`Fig. 3, Figs. 3-5', etc., or `Figure' at the beginning of a
sentence.
All figures, diagrams, etc., must remain within the same area.

Figure captions should be brief and, if possible,
go {\it below} the illustration, e.g. `Fig. 3 A short title'.
They should be typed in point size 9. Very detailed illustrations may
require a full page; if necessary, they may be placed sideways on the
page; when this is done, no text may appear on that page, and the
caption must also read sideways. 

Where possible, figures should be prepared electronically.
We can handle Encapsulated Postscript.

Here is an example of a figure.

\begin{figure}
\label{fig:cc}
\caption{Example of a figure; but it's empty
until you provide an eps (encapsulated postscript)
file. (See the source page.)}
\begin{center}
%\includegraphics[scale=0.8]{test.eps}
%%Uncomment the line above and substitute the filename of your eps file.
\end{center}
\end{figure}

\subsubsection{Tables}

Tables should be referred to in the text as Table 1, Tables 2--7.
They should be centred on the page width, with the table
number, followed by a brief caption in point size 9, typed
{\it above} them if possible. For the
positioning of tables, follow the same rules as those for numbered figures.

\subsection{Appendices}

These should be laid out as the sections in the text,
except that each appendix should start on a new page.
They should be numbered consecutively and be referred
to as Appendix 1, Appendices 1--3, etc. 
Equations, figures and tables should be
quoted as Eq. (A.1.1) and Fig. A.1.1, etc.  

\subsection*{Acknowledgements}

If required, acknowledgements  should appear as a section
immediately before the reference section.
The acknowledgements section should not be numbered.

%\newpage

\section*{APPENDIX 1 -- PREPARING A CONTRIBUTION IN LATEX}

We can accept contributions prepared using Latex.

\section*{TEXTS}

The standard Latex commands \verb|\section, \subsection and \subsubsection|~
should be used for headings.
In the text, leave a blank line after each paragraph.
References to other documents should be
made using the \verb|\cite|~command.
The command \verb|\ref|~should be used to refer to 
equations, figures and tables within your contribution
which should be labelled using the \verb|\label|~command within the 
equation, figure or table environment.

\section*{FORMULAE}

The default Tex format is acceptable for most formulae.
For numbered equations, the construction \\
\verb|\begin{equation}|\\ 
\verb|...|\\
\verb|\label{name}|\\
\verb|\end{equation}|\\
should be used.

\section*{FIGURES}

Figures may be prepared electronically using a variety of
computer applications, as Encapsulated Postscript files.
Make sure that the image, when printed, is of high quality.
To include figures in Latex, a contruction of the form\\
\verb|\begin{figure}[htbp]|\\
\verb|\includegraphics[width=xcm]{file_name.eps}|\\
\verb|\caption{}|\\
\verb|\label{name}|\\
\verb|\end{figure}|\\
should be used.
Figures may be centred
using the \verb|\begin{center}... \end{center}| construct
within the figure environment.

\section*{REFERENCES}

\begin{thebibliography}{99}

\bibitem{Stump}{D. R. Stump, ``How to write a LaTeX paper'',2000.}

\end{thebibliography}

\end{document}


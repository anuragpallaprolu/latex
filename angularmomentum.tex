\documentclass[12pt]{article}
\usepackage{mandi}
\usepackage{graphicsx}
\usepackage{amsmath}
\usepackage{amsfonts}
\begin{document}

\title{Angular Momentum In Quantum Mechanics: A Survey}
\author{Anurag Pallaprolu \\Physics and Electronics\\ BITS Pilani}
\maketitle
\begin{abstract}
This article deals with the angular momentum viewpoint of quantum mechanics. It is a review of the chapters \textbf{Angular Momentum} in both Landau and Lifshitz as well as Devanathan. It also deals with a bit of central force theory and the discussion of the Hydrogen atom as an example of a Coulombic central force.
\end{abstract}
\section*{Introduction}
Any object moving under the influence of a central force exhibits certain characteristic properties. There are a few quantites that do not change when the object is undergoing motion here. Such quantities are called as \textbf{conserved quantities or conserved observables}. According to a standard theorem in classical mechanics known as \textbf{Noether's Theorem} any conservation law is due to some sort of a symmetry in the spacetime structure of the system. In much deeper physics, breaking of such symmetries is believed to create new particles/energy quanta. 

According to Noether's theorem, simply put, any symmetric coordinate or a variable leads to the conservation of its \textbf{canonically conjugate} coordinate or variable. In the case of linear momentum, it is due to the symmetries existant in the x, y and z directions that usually leads to the conservation of momentum in the respective direction. This property of a physical system is called as the \textbf{homogenity of space}. Similarly, in the case of angular momentum, if any angular coordinate shows some symmetry, perhaps the spherical symmetry of the central force field, then the angular momentum is said to be conserved due to the \textbf{isotropy of space.}
\section{Angular Momentum Operator}
Consider any infinitesimal change in a vector, the change being the vector is rotated by an angle of $\delta\phi$ in some plane. The new vector is then given by the simple cross product rule $$\delta\vect{r_{a}} = \vectcrossvect{\delta\vect{\phi}}{\vect{r_{a}}}$$. Now, when such a change in the coordinate is made, then there is a corresponding change in the value of any function $\psi$ represented. This change, if it can be brought about by a unitary operator, then that is our required answer. $$\psi (\vect{r_{i}}+\delta\vect{r_{i}}) = \psi(r_{i}) + \sum_j \delta\vectdotvect{\vect{r_{j}}}{\gradient_{j}\psi}$$
But as we know $\vect{r_{i}}$ representation with a change of angle of $\delta\vect{\phi}$
\end{document}